% Chapter 1 of the Thesis Template File
\chapter{OVERVIEW}

The ever increasing availability of smaller, cheaper, and more power efficient electronic devices mixed with a new type of strain skin sensor makes possible new methods of infrastructure health monitoring of steel beams in bridges.  This new system of custom skins sensors and controlling electronics can assist in the detection of fatigue cracks forming on bridges and other steel infrastructure.

\section{Introduction}

Iowa State University's Civil, Construction, and Environmental Engineering department has developed a new type of strain skin sensor.  To enable the use of this new type of strain sensor requires a new high sensitivity data acquisition and logging system.  This system must be small, energy efficient, and capable of measuring small changes in capacitance in a reasonably short amount of time.  This data acquisition and logging circuitry will ultimately be networked by a wireless sensor network in order to detect fatigue cracks forming on a bridge and assessing the overall health of the bridge during phase two of the project.

\subsection{Capacitive Skin Sensor}

This project revolves around a capacitive strain skin sensor develop by the Iowa State University department of Civil, Construction, and Environmental Engineering under the direction of Dr. Simon Laflamme.  

\subsubsection{Parts of the hypothesis}

Here one particular part of the hypothesis that is 
currently being explained is examined and particular
elements of that part are given careful scrutiny.

% Below \subsubsection
% Sectional commands: \paragraph and \subparagraph may also be used

\subsection{Capacitive Measurement}

Here one particular hypothesis is explained in depth
and is examined in the light of current literature.

\subsubsection{Parts of the second hypothesis}

Here one particular part of the hypothesis that is 
currently being explained is examined and particular
elements of that part are given careful scrutiny.

\subsection{Data Acquisition}

\section{Criteria Review}

Here certain criteria are explained thus eventually
leading to a foregone conclusion.



